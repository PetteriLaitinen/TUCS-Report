\documentclass[12pt]{tucsreport_2013}
% Note, a tucsreport is a specialisation of the LaTeX article class,
% it therefore takes all the same options that the article class does.
% Note also that the "a4paper", "twoside" and "titlepage" options are invoked
% by default in a tucsreport.

\usepackage{times}  % Times fonts

\begin{document}

  \title{Preparing a TUCS Technical Report with {\LaTeX} }

  % Separate multiple authors with "\and".
  % Use the \thanks command to give your address
  \author{Mats Aspn\"as
      \thanks{{\AA}bo Akademi University, Department of Computer Science\\
      Joukahaisenkatu 3-5A, 20520 Turku, Finland\\
      {\tt mats@abo.fi}}
    \and Jonatan Kronqvist\thanks{Turku Centre for Computer Science\\
        Joukahaisenkatu 3-5 A, 20520 Turku, Finland\\
        {\tt jokronqv@abo.fi}}}

  \tucsnumber{0}		% The number of the report (required)
  \isbn{XXX-XXX-XXX-X}	% The ISBN number of the report (reuired)
  \date{March 2013}		% Just give the month and year for the date (required)

  % This extra field gives the TUCS laboratory that produced the report.
  % If you want to specify two laboratorie, separate them with \\
  \lab{TUCS Laboratory}

  % Keywords that describe the report.
  \keywords{TUCS technical reports, \LaTeX}
  \maketitle

  \begin{abstract}
    This document describes how to prepare a TUCS technical report using
      the {\LaTeX} \textsf{tucsreport\_2013} document class.
    This portion of the document was created with the \textsf{abstract}
      environment.
  \end{abstract}


  \section{Preparing Your TUCS Report}

    The \textsf{tucsreport\_2013} document class is just a specialisation of the
      standard {\LaTeX} \cite{latex, companion} \textsf{article} document class.
    To use the \textsf{tucsreport\_2013} document class, just begin your
      {\LaTeX} document like this:
    \begin{verbatim}
      \documentclass[12pt]{tucsreport_2013}
    \end{verbatim}

    You should select either the \textsf{12pt} or \textsf{11pt} options for
    your document, since the printed version of the report will be reduced to
    84 \% of the original size when it is printed on B5 paper. A text size of
    ten points will be too small.

    You can use all the same options with the \textsf{tucsreport\_2013} document
      class that you can use with the \textsf{article} class, and all the same
      commands and packages as well. The only differences between the
      standard \textsf{article} document class and the \textsf{tucsreport\_2013}
      document class are as follows:

    \begin{itemize}
      \item
	The \textsf{a4paper} and \textsf{twoside} options for printing on both sides
	of A4 paper are selected by default by the \textsf{tucsreport\_2013} document
	class. The \textsf{titlepage} option is also selected by default.
      \item
	The \verb"\maketitle" command of a \textsf{tucsreport\_2013} document
	  requires some additional fields to the standard \texttt{author}, \texttt{title},
	  and \texttt{date} fields of the \textsf{article} document class. These fields are:
	
	\begin{description}
	  \item[tucsnumber:]
	    Use the command \verb"\tucsnumber" to set the number of the technical
	    report. If left unset, this field will default to `?'.
	  \item[isbn:]
	    Use the command \verb"\isbn" to set the ISBN number of the technical
	    report. If left unset, this field will default to `?'.
	  \item[keywords:]
	    Use the command \verb"\keywords" to give a list of keywords that
	    characterise the content of the report. The keywords will appear just after
	    the abstract. This field is optional and can be left unset, but it is a good idea
	    to include the keywords.
	  \item[lab:]
	    Use the command \verb"\lab" to set the name of the TUCS  research
	    laboratory that produced the report. The laboratory will will appear after
	    the abstract.
	\end{description}
	
	Instructions for obtaining technical report number and ISBN number
	for a new report can be found on the TUCS' web pages.
	
	Note,
	when setting the \texttt{date} field, just give the month and year.
	For example \verb"\date{August 2013}". If you do not set the date field,
	it will default to the current month and year.
	
      \item
	The last, and most important, difference is that when you use the
	  \textsf{tucsreport\_2013} document class, your document will look like
	  a TUCS Technical Report with front and back pages that obey
	  the graphical recommendations of TUCS.
    \end{itemize}

    For an example of how to prepare a TUCS Technical Report with
    {\LaTeX}, why not take a look at the source code of this document.


  \section{Multiple Authors and Departments}
    The standard {\LaTeX} commands for specifying multiple authors and
      their affiliation are supported. Multiple authors are separated by the
      \verb"\and" command. The \verb"\thanks" command can be used to
      give the affiliation of an author. When required, explicit line breaks
      must be inserted in \verb"\thanks" commands, for instance to put the
      mail address on a separate line.

    Here is an example of an \verb"\author" command for a document with
      multiple authors from different departments.
    {\footnotesize\begin{verbatim}
	\author{
	  Author One\thanks{
	    University of Turku, Department of Information Technology,\\
	    Joukahaisenkatu 3-5, FIN-20520 Turku, Finland \\
	    {\tt author.one@utu.fi} }
	\and
	  Author Two\thanks{
	    {\AA}bo Akademi University, Department of Computer Science,\\
	    Joukahaisenkatu 3-5, FIN-20520 Turku, Finland \\
	    {\tt author.two@abo.fi} }
	}
    \end{verbatim}}


\section{Pictures}

  There are many alternative ways of including pictures in your
  technical report. Both the \textsf{epsfig} and the  \textsf{graphicx} packages
  are known to work well. The following PostScript figure was produced like
  this:

  {\small\begin{verbatim}
	\begin{figure}[ht]
	   \centering
	   \includegraphics[width=5cm]{TUCS_logo.eps}
	   \caption{Example of using a PostScript figure}
	\end{figure}
  \end{verbatim}}

  \begin{figure}[ht]
     \centering
     \includegraphics[width=5cm]{TUCS_logo.eps}
     \caption{Example of using a PostScript figure}
  \end{figure}


\section{Processing Your TUCS Report}


  To process your TUCS report you will need to ensure that {\LaTeX} has
  access to the following files: \\
	\texttt{tucsreport\_2013.cls}, \\
	\texttt{TUCS\_etukansi.eps}, \\
	\texttt{TUCS\_taka\-kansi.eps}, \\
	\texttt{TUCS\_logo.eps}, \\
	\texttt{tylogo.eps}, \\
	\texttt{aalogo.eps}, and \\
	\texttt{tukkk\-logo.eps}. \\
	
    How you arrange that will depend on what sort of system you are running.
    If you are using a UNIX-based system and use \texttt{dvips} as your \textsf{dvi}
    to PostScript driver all you have to do is set your \texttt{TEXINPUTS} path
    to include the directory that contains these files.

\vfill

  \begin{thebibliography}{1}
    \bibitem{latex}
      Leslie Lamport.
      \textit{{\LaTeX}: A document preparation system},
      2nd edition. Addison Wesley. 1994.

    \bibitem{companion}
      Michel Goosens, Frank Mittelbach, Alexander Samarin.
      \textit{The {\LaTeX} Companion}.
      Addison Wesley. 1994.

  \end{thebibliography}

\end{document}
